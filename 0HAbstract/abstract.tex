\chapter*{\centerline{ABSTRACT}}
\addcontentsline{toc}{chapter}{ABSTRACT}
\thispagestyle{plain}
\vspace{-0.5cm}

The voice enables one to understand human emotions, 
which is necessary in enhancing human intelligent interaction. 
In this project, a voice emotion analyzer is trained based on the convolutional 
neural network in order to differentiate between emotional states in response to audio.
 The voice characteristics such as the frequency patterns, energy, and variations in pitch are picked up 
 and fed into the model to generate trusted input. The network is trained to identify happy, sad, angry, fear, disgust, neutral and surprise . 
 Pitch analysis is also used in providing more context in gender of the speaker.According to the obtained results, the model achieves the mean accuracy
of 83.89\% The system is confident in its ability to predict emotions, which allows understanding the behavior and emotional state of the user better.
  The results of the experiments indicate that the given model qualifies as an accurate emotion recognition one in various situations. The sphere of application of this project in the field of mental health checks monitoring,
   interaction between humans and computers, and personalized services in order to be responsive to human emotions.

\par
\textbf{Keywords: Convolutional Neural Networks,  Emotion Recognition, Frequency Patterns Pitch Analysis}